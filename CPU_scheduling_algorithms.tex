\documentclass{article}
\usepackage[utf8]{inputenc}
\usepackage{amsmath}
\usepackage{amssymb}
\usepackage{amsfonts}
\usepackage{textcomp}
\usepackage{kotex}

% \usepackage{algorithmic}
\usepackage[linesnumbered,ruled]{algorithm2e}
\usepackage{epsfig}


\begin{document}



\begin{algorithm} [H]
    \SetAlgoLined
    \KwResult{ Scheduled by first-come, first-served}
    time 0으로 초기화\;
    가장 이른 arrival time을 가진 프로세스들은 ready queue에 들어온다\;
    Time update\;
    \While{infinite}{
      \uIf{ready queue, waiting queue가 다 비었고, 
      terminated queue에 모든 프로세스가 도착했다면}{
      // 1. 스케쥴링 종료\;
      break\;
      }
      \uElseIf{ready queue에 프로세스가 없다면}{
      // 2. Idle 상태\;
      Time update\;
      job queue 또는 waiting queue에서 다음 프로세스를 ready queue로 가져온다\;
      }
      \Else{
      // 3. Running 상태\;
      \underline{ready queue를 arrival time 순으로 정렬}\;
      pop readyQueue 실행하여 crp(currently running process)에 할당\;
      \uIf{I/O가 실행되면}{
      // 3.1 Running → waiting\;
        Time update\;
        job queue 또는 waiting queue에서 다음 프로세스를 ready queue로 가져온다\;
        crp의 cpu burst remaining time update\;
        push waitingQueue(crp)\;
      }
      \Else{
      // 3.2 Running → terminated\;
        Time update\;
        job queue 또는 waiting queue에서 다음 프로세스를 ready queue로 가져온다\;
        crp의 cpu burst remaining time에 0 할당\;
        crp의 turnaround time update\;
        crp의 waiting time update\;
        push terminated Queue(crp)\;
      }
      }
    }
\caption{FCFS}
\end{algorithm}


\begin{algorithm}
    \SetAlgoLined
    \KwResult{ Scheduled by shortest job first Non-preemptively}
    ...\\
      \Else{
      // 3. Running 상태\;
      
      \underline{ready queue를 cpu burst remaining time로 정렬}\;
      pop readyQueue 실행하여 crp(currently running process)에 할당\;
      \uIf{I/O가 실행되면}{
      ...\\
      }
    }
\caption{SJF Non-preemptive}
\end{algorithm}



\begin{algorithm}
    \SetAlgoLined
    \KwResult{ Scheduled by shortest job first preemptively}
    ...\\
      \Else{
      // 3. Running 상태\;
      \underline{ready queue를 cpu burst remaining time로 정렬}\;
      pop readyQueue 실행하여 crp(currently running process)에 할당\;
      \uIf{I/O가 실행되면}{
      // 3.1 Running → waiting\;
        
        \uIf{next time에 crp보다 높은 우선순위의 프로세스가 있으면}{
            \underline{// 3.1.1 running → ready(preempted)}\;
            push readyQueue(crp)\;
            
        }
        \Else{
            Time update\;
            job queue 또는 waiting queue에서 다음 프로세스를 ready queue로 가져온다\;
            crp의 cpu burst remaining time update\;
            push waitingQueue(crp)\;
        }
      }
      \Else{
      // 3.2 Running → terminated\;
        Time update\;
        job queue 또는 waiting queue에서 다음 프로세스를 ready queue로 가져온다\;
        crp의 cpu burst remaining time에 0 할당\;
        crp의 turnaround time update\;
        crp의 waiting time update\;
        push terminated Queue(crp)\;
      }
      }
    }
\caption{SJF Preemptive}
\end{algorithm}







\begin{algorithm}
    \SetAlgoLined
    \KwResult{ Scheduled by shortest job first Non-preemptively}
    ...\\
      \Else{
      // 3. Running 상태\;
      
      \underline{ready queue를 priority 순으로 정렬}\;
      pop readyQueue 실행하여 crp(currently running process)에 할당\;
      \uIf{I/O가 실행되면}{
      ...\\
      }
    }
\caption{Priority Non-preemptive}
\end{algorithm}



\begin{algorithm}
    \SetAlgoLined
    \KwResult{ Scheduled by shortest job first preemptively}
    ...\\
      \Else{
      // 3. Running 상태\;
      \underline{ready queue를 priority순으로 정렬}\;
      pop readyQueue 실행하여 crp(currently running process)에 할당\;
      \uIf{I/O가 실행되면}{
      // 3.1 Running → waiting\;
        
        \uIf{next time에 crp보다 높은 우선순위의 프로세스가 있으면}{
            \underline{// 3.1.1 running → ready(preempted)}\;
            push readyQueue(crp)\;
            
        }
        \Else{
            Time update\;
            job queue 또는 waiting queue에서 다음 프로세스를 ready queue로 가져온다\;
            crp의 cpu burst remaining time update\;
            push waitingQueue(crp)\;
        }
      }
      \Else{
      // 3.2 Running → terminated\;
        Time update\;
        job queue 또는 waiting queue에서 다음 프로세스를 ready queue로 가져온다\;
        crp의 cpu burst remaining time에 0 할당\;
        crp의 turnaround time update\;
        crp의 waiting time update\;
        push terminated Queue(crp)\;
      }
      }
    }
\caption{Priority Preemptive}
\end{algorithm}








\begin{algorithm}
    \SetAlgoLined
    \KwResult{ Scheduled by Round Robin and executed during time quantum}
    ...\\
    \Else{
      // 3. Running 상태\;
      \underline{ready queue를 arrival time 순으로 정렬}\;
      pop readyQueue 실행하여 crp(currently running process)에 할당\;
      \uIf{I/O가 실행되면}{
      // 3.1 Running → waiting\;
        \uIf{\underline{I/O burst start time이 time quantum보다 크면}}{
            I/O burst start time 갱신\;
            CPU burst remaining time 갱신\;
            Time update;
        }
        \Else{
            Time update\;
            job queue 또는 waiting queue에서 다음 프로세스를 ready queue로 가져온다\;
            crp의 cpu burst remaining time update\;
            push waitingQueue(crp)\;
        }
      }
      \Else{
      // 3.2 Running → terminated\;
        \uIf{\underline{CPU burst remaining time이 time quantum보다 크면}}{
            CPU burst remaining time 갱신\;
            Time update;
        }
        \Else{
            Time update\;
            job queue 또는 waiting queue에서 다음 프로세스를 ready queue로 가져온다\;
            crp의 cpu burst remaining time에 0 할당\;
            crp의 turnaround time update\;
            crp의 waiting time update\;
            push terminated Queue(crp)\;
        }
      }
      }
    }
\caption{Round Robin}
\end{algorithm}








\begin{algorithm}
    \SetAlgoLined
    \KwResult{ Scheduled by Lottery and executed during time quantum}
    ...\\
      \Else{
      // 3. Running 상태\;
      \underline{Lottery ticket을 나누어주고, 추첨하여 하나의 프로세스 선정}\;
      pop readyQueue 실행하여 crp(currently running process)에 당첨된 프로세스를 할당\;
      \uIf{I/O가 실행되면}{
      // 3.1 Running → waiting\;
        \uIf{\underline{I/O burst start time이 time quantum보다 크면}}{
            I/O burst start time 갱신\;
            CPU burst remaining time 갱신\;
            Time update;
        }
        \Else{
            Time update\;
            job queue 또는 waiting queue에서 다음 프로세스를 ready queue로 가져온다\;
            crp의 cpu burst remaining time update\;
            push waitingQueue(crp)\;
        }
      }
      \Else{
      // 3.2 Running → terminated\;
        \uIf{\underline{CPU burst remaining time이 time quantum보다 크면}}{
            CPU burst remaining time 갱신\;
            Time update;
        }
        \Else{
            Time update\;
            job queue 또는 waiting queue에서 다음 프로세스를 ready queue로 가져온다\;
            crp의 cpu burst remaining time에 0 할당\;
            crp의 turnaround time update\;
            crp의 waiting time update\;
            push terminated Queue(crp)\;
        }
      }
      }
    }
\caption{Lottery}
\end{algorithm}




\begin{algorithm}
    \SetAlgoLined
    \KwResult{ Scheduled by shortest job first Non-preemptively}
    ...\\
      \Else{
      // 3. Running 상태\;
      \underline{ready queue에 있는 각 프로세스의 DP, 즉 (대기시간 + 실행시간)/실행시간을 구함}\;
      \underline{DP가 가장 높은 프로세스를 ready queue앞에 정렬(Aging)}\;
      pop readyQueue 실행하여 crp(currently running process)에 할당\;
      \uIf{I/O가 실행되면}{
      ...\\
      }
    }
\caption{HRN, Highest Response-Rate Next}
\end{algorithm}






\begin{algorithm}
    \SetAlgoLined
    \KwResult{ Scheduled by using multi-level queue}
    time 0으로 초기화\;
    가장 이른 arrival time을 가진 프로세스들은 ready queue에 들어온다\;
    Time update\;
    \While{infinite}{
      \uIf{\underline{모든 ready queue, waiting queue가 다 비었고,}\ 
      \underline{terminated queue에 모든 프로세스가 도착했다면}}{
      // 1. 스케쥴링 종료\;
      break\;
      }
      \uElseIf{ready queue에 프로세스가 없다면}{
          // 2. Idle 상태\;
          Time update\;
          \underline{job queue 또는 waiting queue에서 다음 프로세스를 우선순위에 맞게 ready queue로 가져온다};
          }
      \uElseIf{\underline{foreground ready queue는 비었고, background에는 프로세스가 있을 때}}{
          // 3. Running 상태(Background)\;
          \underline{ready queue를 arrival time 순으로 정렬}\;
          pop readyQueue 실행하여 crp(currently running process)에 할당\;
          \uIf{I/O가 실행되면}{
          // 3.1 Running → waiting\;
            Time update\;
            \underline{job queue 또는 waiting queue에서 다음 프로세스를 Background ready queue로 가져온다};
            crp의 cpu burst remaining time update\;
            push waitingQueue(crp)\;
          }
          \Else{
          // 3.2 Running → terminated\;
            ...\\
            }
        }
      \Else{
          // 4. Running 상태(foreground)\;
          \underline{ready queue를 arrival time 순으로 정렬}\;
          pop readyQueue 실행하여 crp(currently running process)에 할당\;
          \uIf{I/O가 실행되면}{
          // 4.1 Running → waiting\;
            Time update\;
            \underline{job queue 또는 waiting queue에서 다음 프로세스를 foreground ready queue로 가져온다};
            crp의 cpu burst remaining time update\;
            push waitingQueue(crp)\;
          }
          \Else{
          // 4.2 Running → terminated\;
            ...\\
          }
      }
    }
\caption{Multi-level Queue}
\end{algorithm}



\begin{algorithm}
    \SetAlgoLined
    \KwResult{ Scheduled using multi-level feedback ready queue}
    ...\\
      \uElseIf{\underline{foreground ready queue는 비었고, background에는 프로세스가 있을 때}}{
          // 3. Running 상태(Background)\;
          \underline{대기시간이 aging condition보다 높은 프로세스는}\ 
          \underline{Background에서 Foreground ready queue로 이동한다.}\;
          \underline{ready queue를 arrival time 순으로 정렬}\;
          pop readyQueue 실행하여 crp(currently running process)에 할당\;
          ...\\
      }
      \Else{
          // 4. Running 상태(foreground)\;
          \underline{소모한 CPU burst가 demote condition보다 높은 프로세스는}\ 
          \underline{foreground에서 background ready queue로 이동한다.}\;
          \underline{ready queue를 arrival time 순으로 정렬}\;
          pop readyQueue 실행하여 crp(currently running process)에 할당\;
          ...\\
      }
    }
\caption{Multi-level Feedback Queue}
\end{algorithm}



\begin{algorithm}
    \SetAlgoLined
    \KwResult{ Scheduled by longest i/o first non-preemptively}
    ...\\
      \Else{
      // 3. Running 상태\;
      \underline{ready queue에 I/O를 실행하는 프로세스가 있다면}\ 
      \underline{I/O burst가 긴 프로세스를 앞에서부터 순서대로 정렬}\;
      \underline{ready queue를 cpu burst remaining time로 정렬}\;
      pop readyQueue 실행하여 crp(currently running process)에 할당\;
      \uIf{I/O가 실행되면}{
      ...\\
      }
    }
\caption{LIF Non-preemptive}
\end{algorithm}



\begin{algorithm}
    \SetAlgoLined
    \KwResult{ Scheduled by longest i/o first preemptively}
    ...\\
      \Else{
      // 3. Running 상태\;
      \underline{ready queue에 I/O를 실행하는 프로세스가 있다면}\ 
      \underline{I/O burst가 긴 프로세스를 앞에서부터 순서대로 정렬}
      \underline{ready queue를 cpu burst remaining time로 정렬}\;
      pop readyQueue 실행하여 crp(currently running process)에 할당\;
      \uIf{I/O가 실행되면}{
      // 3.1 Running → waiting\;
        
        \uIf{next time에 crp보다 높은 우선순위의 프로세스가 있으면}{
            \underline{// 3.1.1 running → ready(preempted)}\;
            push readyQueue(crp)\;
            
        }
        \Else{
            Time update\;
            job queue 또는 waiting queue에서 다음 프로세스를 ready queue로 가져온다\;
            crp의 cpu burst remaining time update\;
            push waitingQueue(crp)\;
        }
      }
      \Else{
      // 3.2 Running → terminated\;
        ...\\
      }
      }
    }
\caption{LIF Preemptive}
\end{algorithm}





\end{document}
